\documentclass[11pt]{letter}
\usepackage[margin=1in]{geometry}
\usepackage{hyperref}

\signature{Ian Todd\\Sydney Medical School\\University of Sydney}
\address{Sydney Medical School\\University of Sydney\\Sydney, NSW, Australia\\itod2305@uni.sydney.edu.au}

\begin{document}

\begin{letter}{Editorial Board\\IPI Letters}

\opening{Dear Editors,}

I am pleased to submit ``Time as Information Rate Through Dimensional Apertures: Black Hole Phenomenology from Observer-Relative Channel Capacity'' for consideration in IPI Letters.

This paper proposes that time dilation is fundamentally an information-theoretic phenomenon: the rate of distinguishable state change depends on an observer's \emph{dimensional aperture}---the channel capacity connecting them to underlying dynamics. When the aperture contracts, fewer degrees of freedom are accessible, information accumulation slows, and time dilates.

The key results:
\begin{itemize}
    \item Black hole phenomenology (time freezing, complementarity, area scaling) emerges from observer-relative apertures on high-dimensional dynamics---without invoking general relativity
    \item The thermodynamic cost of aperture contraction (Landauer erasure) spikes at horizons, connecting to Jacobson's thermodynamic derivation of Einstein's equations
    \item Gravitational wave phenomenology (inspiral-merger-ringdown) corresponds to aperture collapse and stabilization dynamics
\end{itemize}

This work extends my paper ``A Thermodynamic Foundation for the Second Law of Infodynamics'' (currently under review at IPI Letters, submitted December 2024). That paper established how maintaining low-dimensional structure requires thermodynamic work; this paper shows how that principle produces time dilation at horizons.

The manuscript is 7 pages with 4 figures. Simulation code is available at \url{https://github.com/todd866/black-hole-aperture}.

I believe this work fits IPI Letters' focus on information-theoretic approaches to fundamental physics. The framework provides a concrete computational demonstration of how observer-relative information access produces relativistic phenomenology.

\closing{Sincerely,}

\end{letter}
\end{document}
