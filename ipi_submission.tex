\documentclass[11pt,english,twoside]{article}
\usepackage{babel}
\usepackage{hyperref}
\usepackage{graphicx}
\usepackage{fancyhdr}
\usepackage{amsmath,amssymb,amsthm}
\usepackage{xcolor}
\usepackage{titlesec}
\usepackage[utf8]{inputenc}
\usepackage{float}
\usepackage{mathtools}
\usepackage{booktabs}
\usepackage[font=scriptsize,labelfont=bf]{caption}
\graphicspath{ {./} {./figures/} }
\titleformat{\section}{\normalsize\bfseries}{\thesection}{1em}{}
\titleformat{\subsection}{\normalsize\bfseries}{\thesubsection}{1em}{}

\newtheorem{definition}{Definition}
\newtheorem{proposition}{Proposition}

\usepackage{geometry}
\geometry{
 a4paper,
 total={170mm,257mm},
 left=20mm,
 right=20mm,
 top=15mm,
 bottom=20mm
}

\begin{document}

\thispagestyle{empty}
\setcounter{page}{1}

\pagestyle{fancy}
\fancyhf{}
\fancyhead{}
\fancyhead[RO,LE]{\vspace{15pt}\\Time as Information Rate Through Dimensional Apertures}
\fancyfoot{}
\fancyfoot[LE,RO]{\thepage}
\fancyfoot[RE,LO]{\url{https://ipipublishing.org/index.php/ipil/}}
\renewcommand{\headrulewidth}{0.4pt}

\begin{minipage}{0.14\textwidth}
\includegraphics[width=0.9\textwidth]{IPI_Pub_Logo.jpg}
\end{minipage}
\hfill
\begin{minipage}{0.5\textwidth}
\includegraphics[width=1.05\textwidth]{IPIL_Logo.jpg}
\end{minipage}
\begin{minipage}{0.3\textwidth}
\begin{flushright}
{\scriptsize
ISSN 2976 - 730X\\
IPI Letters 2025, Vol x (x):x-x\\
\href{https://doi.org/10.59973/ipil.xx}{\color{blue}{https://doi.org/10.59973/ipil.xx}}\\
\medskip
Received: 2025-xx-xx\\
Accepted: 2025-xx-xx\\
Published: 2025-xx-xx\\
}\end{flushright}
\end{minipage}

\vspace{0.5cm}

\par\noindent\rule{\textwidth}{0.5pt}\\
{\color{red}\textbf{Article}}

\begin{center}
\vspace{0.5cm}
{\huge {\bf Time as Information Rate Through Dimensional Apertures: Black Hole Phenomenology from Observer-Relative Channel Capacity}}
\vspace{0.5cm}
\end{center}

\noindent
{\large {\bf Ian Todd$^\bold{1,*}$}}

\vspace{0.1in}

\noindent
{\footnotesize $^1$Sydney Medical School, University of Sydney, Sydney, NSW, Australia

\vspace{0.1in}

\noindent
$^*$Corresponding author: \href{mailto:itod2305@uni.sydney.edu.au}{\color{blue}{itod2305@uni.sydney.edu.au}}
}
\vspace{1cm}

\noindent
{\small {\bf Abstract} - We propose that time dilation is an information-theoretic phenomenon: the rate of distinguishable state change depends on the observer's \emph{dimensional aperture}---the channel capacity connecting them to the underlying dynamics. When the aperture contracts, fewer degrees of freedom are accessible, the rate of information accumulation drops, and time slows. We demonstrate this computationally using coupled oscillators with observer-relative access weights. An ``external'' observer whose aperture closes at a horizon-analogue sees: (1) effective dimension $k_{\mathrm{eff}} \to 2$, (2) correlation rate $\dot{\tau} \to 0$, (3) accessible entropy $S_{\mathrm{acc}}$ dropping, and (4) Landauer erasure cost $Q$ spiking. An ``infalling'' observer with constant aperture sees none of these effects. The same dynamics produce radically different experienced time---black hole complementarity emerges from information geometry. We connect this to gravitational waves: merger/ringdown dynamics correspond to aperture collapse and stabilization. This suggests spacetime geometry may be downstream of dimensional accessibility constraints, consistent with Jacobson's thermodynamic derivation of Einstein's equations.}

\vspace{0.75cm}

\noindent
{\small {\bf Keywords} - Time dilation; Information geometry; Dimensional aperture; Black holes; Landauer principle; Channel capacity; Observer-relative physics}

\vspace{0.2cm}
\par\noindent\rule{\textwidth}{0.5pt}

\section{Introduction}

What is time? Operationally, time is the rate at which distinguishable states succeed one another. A clock ticks because its states change in ways an observer can detect. But detection requires information flow---a channel connecting observer to system. If that channel has limited capacity, the rate of detectable change is bounded.

We propose that \textbf{time dilation is channel contraction}. An observer's \emph{dimensional aperture} $k$ measures how many degrees of freedom they can access. When $k$ drops, fewer correlations can accumulate per unit coordinate time. From the observer's perspective, time slows.

This reframes black hole physics. The external observer's aperture closes at the horizon---not because spacetime ends, but because the information channel to interior degrees of freedom is squeezed. The infalling observer maintains full access and experiences normal time. Same dynamics, different channels, different clocks.

The framework connects to Jacobson's thermodynamic derivation of Einstein's equations [1]: horizons are where the thermodynamic cost of maintaining a coarse-grained description becomes extreme. Our simulation makes this operational: the Landauer cost of aperture compression spikes at the horizon.

This paper provides a computational demonstration, not a derivation. We show that observer-relative dimensional apertures on high-dimensional dynamics \emph{reproduce} black hole phenomenology---time freezing, complementarity, area scaling, thermodynamic costs---without invoking general relativity.

\section{Framework}

\subsection{Dimensional Aperture}

Consider a system with $N$ coupled degrees of freedom and state $\mathbf{x} \in \mathbb{R}^{2N}$. Different observers have different access.

\begin{definition}[Aperture]
An observer's aperture is a weight vector $\mathbf{w} \in [0,1]^N$ determining access to each mode:
\begin{equation}
    \mathbf{x}_{\mathrm{obs}} = \mathbf{w} \odot \mathbf{x}
\end{equation}
\end{definition}

We distinguish two measures of effective dimension:

\textbf{Channel participation} (from weights alone):
\begin{equation}
    k_{w} = \frac{(\sum_i w_i)^2}{\sum_i w_i^2}
\end{equation}

\textbf{Dynamical dimension} (from observed state statistics):
\begin{equation}
    k_{\mathrm{dyn}} = \frac{(\sum_j \lambda_j)^2}{\sum_j \lambda_j^2}
\end{equation}
where $\{\lambda_j\}$ are eigenvalues of the observed covariance $C_{\mathrm{obs}}$.

For a uniform aperture ($w_i = 1$), $k_w = N$. As weights concentrate on fewer modes, $k_w$ drops. The key prediction is that $k_{\mathrm{dyn}}$ tracks $k_w$: aperture squeezing reduces the dynamical complexity of the observed state, not just the number of accessible channels.

\subsection{Time as Information Rate}

We treat the aperture weights as defining an induced metric $G = \mathrm{diag}(w_1, \ldots, w_N)$ on the observer's state space. The correlation accumulation rate is then the geodesic speed:
\begin{equation}
    \dot{\tau} = \sqrt{\sum_i w_i \dot{x}_i^2} = \sqrt{\dot{\mathbf{x}}^T G \dot{\mathbf{x}}}
\end{equation}

This measures how fast the observer's accessible state changes. When $\mathbf{w} \to 0$, the metric degenerates and $\dot{\tau} \to 0$: time stops.

This connects to information geometry [2]: the Fisher metric determines the maximum rate of distinguishable state change. Contracting the aperture contracts the accessible Fisher volume, reducing geodesic speed through state space. Time slows because fewer distinguishable states can be traversed per unit coordinate time.

\subsection{Thermodynamic Cost}

When the aperture squeezes, degrees of freedom are traced out---information is erased. Landauer's principle [3] requires minimum heat dissipation:
\begin{equation}
    Q_{\mathrm{heat}} \geq k_{\mathrm{B}}T \ln 2 \cdot \Delta I_{\mathrm{erased}}
\end{equation}
where $\Delta I$ is measured in bits.

We track a \textbf{Landauer proxy} (in nats, dimensionless):
\begin{equation}
    S_{\mathrm{acc}} = \frac{1}{2}\log\det (C_{\mathrm{obs}} + \epsilon I)
\end{equation}
where $C_{\mathrm{obs}}$ is the covariance of accessible states and $\epsilon$ is a regularization constant. This is a log-volume measure (differential entropy up to a constant), not Shannon entropy; it can be negative.

The cumulative erasure proxy:
\begin{equation}
    Q(t) = \sum_{s \leq t} \max(0, S_{\mathrm{acc}}(s-1) - S_{\mathrm{acc}}(s))
\end{equation}
To obtain physical heat, multiply by $k_{\mathrm{B}}T$ and convert units.

This connects to [4]: maintaining a low-dimensional description of a high-dimensional system requires work. Horizons are where this work diverges.

\section{Simulation}

\subsection{Setup}

We simulate $N = 50$ coupled harmonic oscillators:
\begin{equation}
    H = \frac{1}{2}\sum_i p_i^2 + \frac{1}{2}\sum_i x_i^2 + \frac{\kappa}{2}\sum_{\langle ij \rangle}(x_i - x_j)^2
\end{equation}

Two observers watch the same dynamics:

\textbf{External observer}: Aperture depends on ``radius'' $r \in (0, 1]$:
\begin{equation}
    w_i(r) = r^{3f_i}, \quad f_i = i/N
\end{equation}
High-frequency modes are suppressed faster as $r \to 0$ (approaching horizon).

\textbf{Infalling observer}: $w_i = 1$ always (full access).

\subsection{Results}

Figure~\ref{fig:time_dilation} shows the key result. As the external observer's radius decreases:
\begin{itemize}
    \item Both $k_w$ and $k_{\mathrm{dyn}}$ drop (channel participation tracks dynamical dimension)
    \item $\dot{\tau}$ drops proportionally
    \item Accumulated proper time $\tau$ asymptotes
\end{itemize}

The infalling observer sees constant $k_{\mathrm{dyn}}$, constant $\dot{\tau}$, linear $\tau$. Same dynamics, different apertures, different clocks.

\begin{figure}[H]
    \centering
    \includegraphics[width=\textwidth]{fig1_time_dilation.png}
    \caption{Time dilation from aperture contraction. External observer (red) sees dimensional collapse and time freezing; infalling observer (cyan) maintains constant flow. This is black hole complementarity without GR.}
    \label{fig:time_dilation}
\end{figure}

Figure~\ref{fig:thermodynamics} shows the thermodynamic structure. As the aperture squeezes:
\begin{itemize}
    \item $S_{\mathrm{acc}}$ drops (degrees of freedom traced out)
    \item $Q$ spikes (Landauer cost of erasure)
\end{itemize}

The horizon is where maintaining a description becomes thermodynamically expensive---Jacobson's insight made computational.

\begin{figure}[H]
    \centering
    \includegraphics[width=\textwidth]{fig2_thermodynamics.png}
    \caption{Thermodynamic cost of aperture squeezing. External observer pays increasing Landauer cost; infalling observer pays none.}
    \label{fig:thermodynamics}
\end{figure}

\subsection{Connection to Gravitational Waves}

Figure~\ref{fig:ligo} shows a simulated merger. The aperture dynamics produce:
\begin{itemize}
    \item \textbf{Inspiral}: gradual aperture contraction
    \item \textbf{Merger}: rapid dimensional collapse
    \item \textbf{Ringdown}: damped oscillations as aperture stabilizes
\end{itemize}

If spacetime geometry reflects information accessibility, then gravitational waves are oscillations in channel capacity. The ringdown frequency would encode how fast the merged system's aperture stabilizes.

\begin{figure}[H]
    \centering
    \includegraphics[width=\textwidth]{fig3_ligo_connection.png}
    \caption{Merger simulation. Aperture perturbations produce inspiral-merger-ringdown structure with qualitative similarity to LIGO waveforms [5]. This is a shape comparison; we do not derive GR predictions.}
    \label{fig:ligo}
\end{figure}

\subsection{Comparison to Schwarzschild}

Figure~\ref{fig:schwarzschild} compares our aperture-based $k_{\mathrm{dyn}}/k_{\mathrm{max}}$ to the Schwarzschild time dilation factor $\sqrt{1-r_s/r}$. The qualitative behavior matches: both approach zero at the horizon, both are smooth elsewhere. This is a shape comparison only; we do not derive the Schwarzschild factor, and many monotone functions share this qualitative behavior.

\begin{figure}[H]
    \centering
    \includegraphics[width=0.7\textwidth]{fig4_k_vs_radius.png}
    \caption{Dimensional collapse (solid) vs Schwarzschild time dilation (dashed). Qualitative similarity; not a derivation.}
    \label{fig:schwarzschild}
\end{figure}

\section{Discussion}

\subsection{What This Demonstrates}

We have shown that \textbf{horizon-like phenomenology} emerges from observer-relative dimensional apertures on high-dimensional dynamics.

Without invoking Einstein's equations, we recovered:
\begin{itemize}
    \item Time freezing at a boundary (horizon-analogue)
    \item Observer-dependent experienced time (complementarity-like)
    \item Boundary-dominated access (area-like scaling)
    \item Thermodynamic costs spiking at boundaries
    \item Ringdown-like dynamics after perturbation
\end{itemize}

The key observation: we defined the aperture constraint, but the \emph{consequences}---thermodynamics, time dilation, complementarity---emerged naturally without being programmed in.

\subsection{Interpretation}

This is a computational demonstration, not a claim that spacetime \emph{is} information geometry. But if Jacobson [1] is right that Einstein's equations follow from thermodynamic consistency at horizons, then our simulation provides a substrate where this is manifest: the horizon is where aperture compression cost diverges.

The relationship is:
\begin{center}
\emph{Aperture contraction} $\to$ \emph{Information erasure} $\to$ \emph{Thermodynamic cost} $\to$ \emph{Time dilation}
\end{center}

GR produces aperture contraction via spacetime curvature. But the aperture-time relationship may be more fundamental.

\subsection{Connection to Infodynamics}

This framework connects to the second law of infodynamics [6]. As apertures contract, structure-information decreases---the observer's accessible description becomes simpler. The thermodynamic cost of maintaining complex descriptions at horizons drives evolution toward symmetric, high-entropy states.

Black holes are endpoints of this evolution: maximum thermodynamic cost, minimum accessible structure-information, time frozen.

\subsection{Limitations}

This is a toy model. We have not derived the aperture-radius relationship from first principles, shown quantitative agreement with GR, or addressed the information paradox directly. The claim is modest: the phenomenology \emph{can} emerge from aperture constraints.

\section{Conclusion}

Time dilation is what channel contraction looks like from inside. When an observer's dimensional aperture closes, their information rate drops, and time slows. Black hole phenomenology---complementarity, horizon freezing, thermodynamic costs---emerges from this single principle without invoking general relativity.

This suggests a research program: derive spacetime geometry from information-geometric constraints on observer accessibility. The mathematics is substrate-agnostic. If it works for coupled oscillators, it may work for spacetime.

\vspace{0.5cm}
\noindent\textbf{Code availability:} \url{https://github.com/todd866/black-hole-aperture}

\vspace{0.5cm}
\noindent\textbf{Acknowledgments:} This work was developed in conversation with Claude (Anthropic).

\vspace{0.5cm}
\noindent\textbf{References}

\begin{enumerate}
\item T. Jacobson, ``Thermodynamics of spacetime: the Einstein equation of state,'' \textit{Phys. Rev. Lett.} \textbf{75}, 1260 (1995).

\item S. Amari and H. Nagaoka, \textit{Methods of Information Geometry} (AMS, 2000).

\item R. Landauer, ``Irreversibility and heat generation in the computing process,'' \textit{IBM J. Res. Dev.} \textbf{5}, 183 (1961).

\item I. Todd, ``A Thermodynamic Foundation for the Second Law of Infodynamics,'' \textit{IPI Letters} (2025, under review).

\item B. P. Abbott \textit{et al.}, ``Observation of gravitational waves from a binary black hole merger,'' \textit{Phys. Rev. Lett.} \textbf{116}, 061102 (2016).

\item M. M. Vopson, ``The second law of infodynamics and its implications for the simulated universe hypothesis,'' \textit{AIP Advances} \textbf{13}, 105308 (2023).
\end{enumerate}

\end{document}
